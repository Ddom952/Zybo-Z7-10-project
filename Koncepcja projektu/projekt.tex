\documentclass[a4paper, 12pt]{article}
\usepackage[utf8]{inputenc}
\usepackage{geometry}
\usepackage{polski}
\usepackage{graphicx}
\usepackage{float}
\usepackage{etoolbox,refcount}
\usepackage{multicol}
\usepackage{multirow}
\usepackage{fancyhdr}
\usepackage{listings}
\usepackage{array}
\usepackage{textcomp}
\usepackage{listings}
\usepackage{graphicx}
\usepackage{caption}
\usepackage{subcaption}
\usepackage{color}
\usepackage{hyperref}

\usepackage{tikz}
\usetikzlibrary{calc,through,backgrounds,positioning}

\definecolor{mygreen}{rgb}{0,0.6,0}
\definecolor{mygray}{rgb}{0.5,0.5,0.5}
\definecolor{mymauve}{rgb}{0.58,0,0.82}

\lstset{ %
	backgroundcolor=\color{white},   % choose the background color; you must add \usepackage{color} or \usepackage{xcolor}; should come as last argument
	basicstyle=\tiny,        % the size of the fonts that are used for the code
	breakatwhitespace=false,         % sets if automatic breaks should only happen at whitespace
	breaklines=true,                 % sets automatic line breaking
	captionpos=b,                    % sets the caption-position to bottom
	commentstyle=\color{mygreen},    % comment style
	deletekeywords={...},            % if you want to delete keywords from the given language
	escapeinside={\%*}{*)},          % if you want to add LaTeX within your code
	extendedchars=true,              % lets you use non-ASCII characters; for 8-bits encodings only, does not work with UTF-8
	frame=single,	                   % adds a frame around the code
	keepspaces=true,                 % keeps spaces in text, useful for keeping indentation of code (possibly needs columns=flexible)
	keywordstyle=\color{blue},       % keyword style
	language=Octave,                 % the language of the code
	morekeywords={*,...},           % if you want to add more keywords to the set
	numbers=left,                    % where to put the line-numbers; possible values are (none, left, right)
	numbersep=5pt,                   % how far the line-numbers are from the code
	numberstyle=\tiny\color{mygray}, % the style that is used for the line-numbers
	rulecolor=\color{black},         % if not set, the frame-color may be changed on line-breaks within not-black text (e.g. comments (green here))
	showspaces=false,                % show spaces everywhere adding particular underscores; it overrides 'showstringspaces'
	showstringspaces=false,          % underline spaces within strings only
	showtabs=false,                  % show tabs within strings adding particular underscores
	stepnumber=2,                    % the step between two line-numbers. If it's 1, each line will be numbered
	stringstyle=\color{mymauve},     % string literal style
	tabsize=1,	                   % sets default tabsize to 2 spaces
	title=\lstname                   % show the filename of files included with \lstinputlisting; also try caption instead of title
}

\newcolumntype{L}[1]{>{\raggedright\let\newline\\\arraybackslash\hspace{0pt}}m{#1}}
\newcolumntype{C}[1]{>{\centering\let\newline\\\arraybackslash\hspace{0pt}}m{#1}}
\newcolumntype{R}[1]{>{\raggedleft\let\newline\\\arraybackslash\hspace{0pt}}m{#1}}
\title{\Huge{Koncepcja projektu na przedmiot ZMPSR: Obsługa kodeka audio na ZyboZ7-10}}
\author{Dominik Baljon, Marcin Śladowski}
\date{}
\newgeometry{left=2.5cm, right=2.5cm, bottom=2.75cm, top=2.75cm}

\begin{document}
\maketitle

\section{Wprowadzenie}
W ramach projektu powstał pomysł zrealizowania obsługi kodeka audio dostępnego na Zybo, w celu umożliwienia odtwarzania ścieżek audio znajdujących się na karcie SD. Możliwe rozwinięcia (po zaimplementowaniu podstawowej funkcjonalności) o nagrywanie dźwięku, obliczania widma częstotliwościowego wraz z jego prezentacją (LED-y, przesłanie po UART w celu prezentacji).
Kodek znajdujący się na płytce drukowanej to SSM2603 firmy \textit{Analog Devices}. Powyższy kodek posiada interfejsy I2C oraz I2S oraz wejście zegara.

\section{Podział zadań}
Znakiem * zostały oznaczone zadania w przypadku rozwinięcia projektu.
\subsection{Strona FPGA}
\begin{itemize}
	\item wysyłanie próbek audio do kodeka przy wykorzystaniu I2S. Możliwe rozwinięcie do odbierania próbek
	\item komunikacja UART ze zdalną stacją umożliwiająca ustawienie parametrów odtwarzania, nagrywania ścieżek audio
	\item konfiguracja kodeka przy pomocy interfejsu I2C
	\item odczyt próbek audio z pamięci (DMA) po umieszczeniu ich tam przez procesor. Ewentualny zapis próbek do pamięci przy nagrywaniu. Konieczna odpowiednia synchronizacja z częścią procesorową
	\item generacja sygnału zegarowego dla kodeka audio
	\item odbiór informacji o ścieżkach audio znajdujących się na karcie SD oraz przesłanie ich przy pomocy UART-a
	\item odbiór i odpowiednia reakcja na komendy \textit{play}, \textit{pause}, \textit{stop}, \textit{next}, \textit{previous} przesyłane poprzez UART
	\item[*] obliczenie widma częstotliwościowego dla odtwarzanych bądź nagrywanych ścieżek audio
	\item[*] przestawienie wyników obliczenia widma częstotliwościowego na Led-ch bądź przesyłanie przy pomocy UART-a do PC, który zaprezentuje wyniki
\end{itemize}

\subsection{Strona procesora}
\begin{itemize}
	\item odczyt plików audio z karty SD, ewentualne przetworzenie do odpowiedniego formatu
	\item umieszczenie próbek audio w pamięci dostępnej dla części FPGA. Konieczna odpowiednia synchronizacja z częścią procesorową.
	\item odbiór informacji od części FPGA w ramach zmian ścieżek dźwiękowych, zatrzymać itd.
\end{itemize}

Na poniższym rysunku przedstawiono schemat działania. Przerywanymi liniami zaznaczone elementy w przypadku rozszerzenia.

\begin{figure}[!ht]
	\centerline{\includegraphics[scale=1]{img/ZYBO_block_diagram2.pdf}}
	\caption{Schemat działania}
\end{figure}

\section{Podsumowanie}
Realizacja takiego projektu pozwoli na lepsze zrozumienie przetwarzania audio, synchronizacji częściami układów Zynq-7000 ARM/FPGA oraz zarządzania, dostępem do pamięci.



\end{document}